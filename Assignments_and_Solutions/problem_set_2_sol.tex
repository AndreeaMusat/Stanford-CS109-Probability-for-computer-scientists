\documentclass[10pt,a4paper,oneside,draft]{report}
\usepackage[utf8]{inputenc}
\usepackage{amsmath,amssymb}
\usepackage{mathtools}

\DeclareRobustCommand{\bbone}{\text{\usefont{U}{bbold}{m}{n}1}}

\DeclareMathOperator{\EX}{\mathbb{E}}% expected value
\DeclarePairedDelimiter\ceil{\lceil}{\rceil}
\DeclarePairedDelimiter\floor{\lfloor}{\rfloor}

\title{CS109 assignment 2}
\author{andreea.a.musat@gmail.com}
\date{January 2019}

\usepackage{natbib}
\usepackage{graphicx}
\usepackage{xcolor}

\newcommand\myworries[1]{\textcolor{red}{#1}}

\begin{document}

\maketitle

\textbf{Warmup} \\

\textbf{1.} Say in Silicon Valley, 36\% of engineers program in Java and 24\% of the engineers who
program in Java also program in C++. Furthermore, 33\% of engineers program in C++. \\
a. What is the probability that a randomly selected engineer programs in Java and C++? \\
b. What is the conditional probability that a randomly selected engineer programs in Java
given that he/she programs in C++? \\

\textbf{Solution} \\

a. $P(J, C) = P(J) * P(C|J) = 0.36 * 0.24 = 0.0864$ \\

b. $P(J|C) = \frac{ {P(C|J) P(J)} }{P(C)} = \frac{ {0.24 * 0.36} }{0.33} = 0.2618$ \\

\textbf{2.} Two cards are randomly chosen without replacement from an ordinary deck of 52 cards. Let E be the event that both cards are Aces. Let F be the event that the first card chosen is the Ace of Spades. Compute $P(E|F)$.

\textbf{Solution} \\

Intuitively, we'd say that if we already know that the first card is Ace of Spades, there are 51 cards left to choose from and 3 of those would make event E happen, so the probability is $\frac{3}{51}$. We can check this result if we apply Bayes' rule: $P(E|F) = \frac{ {P(F|E) P(E)} } {P(F)}$. We know that $P(F) = \frac{1}{52}$, as there is exactly one Ace of Spades in the deck. Also, $P(E) = P(\textnormal{first card is Ace}) P(\textnormal{second card is Ace}) = \frac{4}{52} \frac{3}{51}$. Given that we have a pair of Aces, the probability that first one in the pair is the Ace of Spades is $\frac{1}{4} = P(F|E)$. Replacing the values in Bayes' rule above, we get the same result. \\

\textbf{3.} Five servers are located in a computer cluster. After one year, each server independently is still working with probability p, and otherwise fails (with probability 1 – p). \\
a. What is the probability that at least 1 server is still working after one year? \\
b. What is the probability that exactly 3 servers are still working after one year? \\
c. What is the probability that at least 3 servers are still working after one year? \\

\textbf{Solution} \\

a. $P(\# \textnormal{working clusters} \geq 1) = 1 - P(\# \textnormal{working clusters} = 0) = 1 - (1-p)^5.$

b. $P(\# \textnormal{working clusters} = 3) = {5 \choose 3} {p^3} {(1-p)^2} = 10 {p^3} {(1-p)^2} $\\

c. \begin{center}
$P(\# \textnormal{working clusters} >= 3) =
P(\# \textnormal{working clusters} = 3) + P(\# \textnormal{working clusters} = 4) + P(\# \textnormal{working clusters} = 5) = {5 \choose 3} {p^3} {(1-p)^2} + {5 \choose 4} {p^4} {(1-p)^1} + {5 \choose 5} {p^5} {(1-p)^0} $
\end{center}

\textbf{4.} A website wants to detect if a visitor is a robot. They give the visitor three CAPTCHA tests that are hard for robots, but easy for humans. If the visitor fails in one of the tests, they are flagged as a robot. The probability that a human succeeds at a single test is 0.95, while a robot only succeeds with probability 0.3. Assume all tests are independent. \\
a. If a robot visits the website, what is the probability they get flagged? \\
b. If a visitor is human, what is the probability they get flagged? \\
c. The fraction of visitors on the site that are robots is 1/10. Suppose a visitor gets flagged. What is the probability that visitor is a robot? \\

\textbf{Solution} \\

a. $P(\textnormal{R is flagged}) = 1 - P(\textnormal{R gets all 3 tests correct}) = 1 - 0.3^3 = 0.973$.\\

b. $P(\textnormal{H is flagged}) = 1 - P(\textnormal{H gets all 3 tests correct}) = 1 - 0.95^3 = 0.1426$.\\

c. $P(\textnormal{V=R}|\textnormal{V is flagged}) = \frac{ {P(\textnormal{V is flagged} | \textnormal{V=R}) P(\textnormal{V=R})} }{ P(\textnormal{V is flagged}) } 
= \\
\frac{ {P(\textnormal{R is flagged}) P(\textnormal{V=R})} }{ {P(\textnormal{R is flagged})P(\textnormal{V=R}) + P(\textnormal{H is flagged})P(\textnormal{V=H})} } 
= \frac{ {0.973 * 0.1} }{ {0.973 * 0.1 + 0.1426 * 0.9} } = 0.4312
$\\

	
\textbf{5.} Recall the game set-up in the “St. Petersburg’s paradox” discussed in class: there is a fair coin which comes up "heads" with probability p = 0.5. The coin is flipped repeatedly until the first "tails" appears. Let N = number of coin flips before the first "tails" appears (i.e., N = the number of consecutive "heads" that appear). Given that no one really has infinite money to offer as payoff for the game, consider a variant of the game where you win MIN$(2^N , X)$,
where X is the maximum amount that the game provider will pay you after playing. Compute the expected payoff of the game for the following values of X. Show your work.\\
a. X = \$5. \\
b. X = \$500. \\
c. X = \$4096. \\

\textbf{Solution} \\

In the most general way, we can write: \\



\begin{gather*}
\EX[\textnormal{payoff}] = \sum_{i=0}^{i=\infty} (\frac{1}{2})^i \frac{1}{2} \min(2^i, X) = \sum_{i=0}^{i=\floor*{\log_2 X}} (\frac{1}{2^{i + 1}}) 2^i + \sum_{i>\floor*{\log_2 X}} \frac{1}{2^{i + 1}} X = \\
\frac{1}{2} (1 + \floor*{\log_2 X}) + \frac{1}{2^{ \floor*{\log_2 X} + 1 }} X \sum_{i>0} \frac{1}{2 ^ i} = \frac{1}{2} (1 + \floor*{\log_2 X}) + \frac{X}{2^{ \floor*{\log_2 X} }}
\end{gather*}

a. Using X=\$5 in the equation above, we get: $\EX[\textnormal{payoff}] = \frac{3}{2} + \frac{5}{4} = 2.$ \\

b. $\EX[\textnormal{payoff} | X=\$500] = \frac{9}{2} + \frac{500}{256} = 5.$ \\

c. $\EX[\textnormal{payoff} | X=\$4096] = \frac{13}{2} + \frac{4096}{4096} = 7.$\\

\textbf{6.} A bit string of length n is generated randomly such that each bit is generated independently
with probability p that the bit is a 1 (and 0 otherwise). How large does n need to be (in terms
of p) so that the probability that there is at least one 1 in the string is at least 0.7? \\

\textbf{Solution} \\

$P(\#_1(string) \geq 1) = 1 - P(\#_1(string)=0) = 1 - (1 - p)^n$.
Thus, $P(\#_1(string) \geq 1) \geq 0.7$ is equivalent to:
$1 - (1 - p)^n \geq 0.7 \iff 0.3 \geq (1 - p)^n \iff \log_{1-p} 0.3 \leq n $ 

\end{document}

